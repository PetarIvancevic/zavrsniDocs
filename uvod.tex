\section{Uvod}
Tema ovog završnog rada je izrada 3D igre u unity-u. Predznanje koje je potrebno za napraviti nešto slično ovom radu su C\#, objektno programiranje, osnovno poznavanja fizike, te naravno kako voziti automobil. Kroz rad se postepeno upoznaje sa unity okruženjem i objašnjavaju najvažnije funkcionalnosti. Motivacija za ovakav tip rada je autorova želja za stvaranjem nečega što će i on sam moći koristiti, te graditi i razvijati kroz duži period.

U drugom poglavlju su navedene sve tehnologije koje su korištene za realizaciju igre. SketchUp i Unity su ukratko objašnjeni, kako bi se znalo odakle dolaze modeli i nekakve osnove.

Treće poglavlje detaljnije opisuje Unity od osnovnih komponenti do načina na koji se pokreće vozilo unutar igre. Najvažniji elementi su navedeni, te nakon čitanja ovog poglavlja bi čitatelj mogao dobiti sliku o kompleksnosti alata.

Četvrto poglavlje objašnjava skriptiranje unutar Unity platforme. Opisuje se što je skriptiranje, kako se izrađuju skripte, osnovni sadržaj skripti, te neki tipovi funkcija korištenih u igri. Postoji još mnogo više funkcija koje je moguće vidjeti u dokumentaciji Unity platforme, a ovdje su navedene najčešće korištene. 

Peto poglavlje objašnjava izradu korisničkog sučelja, te što je korisničko sučelje ustvari. Objašnjava kako se izrađuje, što je platno, kako napraviti izbornike, te manipulirati tekstovima preko referenci.

Šesto poglavlje objašnjava kako se koristi zvuk unutar Unity platforme, na koji način se ustvari proizvodi zvuk, te neke od glavnih komponenti zvuka. 

Sedmo poglavlje sadrži opis klasa koje su korištene unutar igre. Opisuje sve podatkovne članove i funkcije, odnosno metode.

Zadnje poglavlje je zaključak gdje se daje osvrt na rad, te osvrt autora nakon izrade igre.