\section{Uvod}
Tema ovog završnog rada je izrada 3D igre u unity-u. Predznanje koje je potrebno za napraviti nešto slično ovom radu su C\#, objektno orijentiranog programiranja, osnovno poznavanja fizike, te naravno kako voziti automobil. Kroz rad se postepeno upoznaje sa unity okruženjem i objašnjavaju najvažnije funkcionalnosti kako su realizirane. Motivacija za ovakav tip rada je zbog autorove želje za stvaranjem nečega što će i on sam moći koristiti, te graditi i razvijati kroz duži period.  \par

U drugom poglavlju su navedene sve tehnologije koje su korištene za realizaciju igre. SketchUp su ukratko objašnjen, kako bi se znalo odakle dolaze modeli i nekakve osnove. \par

Treće poglavlje detaljnije opisuje unity, od osnovnih komponenti, do načina na koji se pokreće vozilo unutar igre. Najvažniji elementi su navedeni, te nakon čitanja ovog poglavlja bi čitatelj mogao dobiti sliku o kompleksnosti alata. \par

Četvrto poglavlje objašnjava skriptiranje unutar unity-a. Opisuje se što je skriptiranje, kako se izrađuju, osnovni sadržaj skripti, te neke tipove funkcija korištene u igri. Postoji još mnogo više funkcija koje je moguće vidjeti u dokumentaciji od unity-a, a ovdje su navedene najčešće korištene. \par

Peto poglavlje objašnjava izradu korisničkog sučelja, te što je korisničko sučelje usvari. Objašnjava kako se izrađuje, što je platno, kako napraviti izbornike, te manipulrati tekstovima preko referenci. \par

Šesto poglavlje objašnjava kako se koristi zvuk unutar unity-a. Na koji način se ustvari apstraktira zvuk, te neke od glavnih komponenti zvuka. \par

Sedmo poglavlje sadrži opis klasa koje su korištene unutar igre. Opisuje sve podatkovne članove i funkcije, odnosno metode. \par

Zadnje poglavlje je zaključak gdje se daje osvrt na rad, te misli autora nakon izrade igre.