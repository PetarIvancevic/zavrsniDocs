\section{Zaključak}
Prvih tjedan dana bilo koje tehnologije programerima je najteži period i u tom vremenu će se moći vidjeti hoće li naučiti raditi s tom tehnologijom ili ne. Upoznavanje sa Unity platformom je veoma lako zbog brojnih video zapisa u kojima je detaljno objašnjen svaki dio. Ukoliko postoji želja za učenjem ove tehnologije i ustrajnost od godinu dana, programer će moći napraviti koju god igru poželi.

Za naučiti raditi igre u Unity platforme, a i za programiranje nije dovoljno samo znati programirati, već i imati želju za stvaranjem. To bi trebala biti nekakva umjetnost gdje se svaki problem može riješiti na više načina, a na programeru je da svojim stilom dođe do najboljeg. Proces izrade igre je veoma kompleksan ako se radi o većoj igri. Osobe koje vole znati \emph{kako} nešto funkcionira bi trebali biti programeri jer su upravo programeri ti koji kada apstrahtiraju nešto moraju poznavati i kako to funkcionira.
 
Istina je da programeri ne prestaju učiti jer se tehnologija uvijek mijenja, a ako rade različite tipove programa, moraju poznavati taj novi sustav kako bi mogli apstrahtirati njegove elemente. Nakon toga i isprogramirati.