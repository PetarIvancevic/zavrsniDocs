\section{Korisničko sučelje}
Svaka igra mora imati svoje korisničko sučelje (\emph{eng.~User Interface, UI}) gdje će korisnici moći mijenjati pojedine opcije igre. Unity za ovo pruža poseban igrajući objekt koji se zove platno (\emph{eng.~Canvas}). Ukoliko platno već nije izrađeno, onda prilikom izrade teksta za sučelje će Unity sam napraviti i platno i tekst kao dijete platna. Dakle sve objekte koji pripadaju korisničkom sučelju moraju biti dijete platna u hijerarhiji.
\subsection{Platno}
Sva platna su ustvari 2D objekti kojima je dodana dubina zbog prikaza u 3D prostoru. Prebacivanjem u 3D pogled će se moći vidjeti da platna izgledaju kao običan list papira na kojima je nekakav tekst, botun ili slika. Na slici \ref{fig:primjer_platna} se može vidjeti kako izgleda platno u 3D svijetu. Igrajući objekti koji su dio platna će se prikazivati prema njihovom položaju u hijerarhiji. Dakle ako imamo više igrajućih objekata, onda će se prvo nacrtati prvo dijete i svi igrajući objekti koji su njegova djeca, zatim drugo dijete i tako dalje. Platno također ima različite modove izrade (\emph{eng.~Render modes}).
\begin{figure}[h]
	\includegraphics[width=12.5cm, height=8cm]{primjer_papir_platna.png}
	\centering
	\caption{Primjer platna u 3D svijetu}
	\label{fig:primjer_platna}
\end{figure}

\subsubsection{Raspored elemenata}
Pod rasporedom (\emph{eng.~Layout}) se misli pozicija pojedinih elemenata unutar platna, dakle dimenzije i odnos prema drugim elementima. Sidrišta (\emph{eng.~Anchors}) imaju jako veliku ulogu pri odnosu pojedinih elemata, a o njima će više biti rečeno u nastavaku. Veličina platna se može mijenjati korištenjem alata \emph{Rect Transform}. Uglavnom se koristi za 2D objekte, ali naravno može i za 3D objekte. Isto tako se može mijenjati pozicija elemenata označavanjem i povlačenjem na željenu lokaciju. Potrebno je razlikovati, prilikom mijenjanja veličine objekta, da postoje dvije različite transformacije koje se čine jednake. Ako se koristi alat \emph{Rect Tool} označen objekt tada mijenja svoju veličinu (\emph{eng.~Scale}), a ako je selektiran \emph{Rect Transform} mijenjaju se dimenzije objekta (širina i visina). 

\subsubsection{Sidrišta}
Sidrišta su četiri trokutasta elementa koja definiraju točku od koje će se kretati elementi i koju će pratiti za promjene. Zanimljiva funkcionalnost koju omogućavaje je da element mijenja dimenzije ovisno o elemntu kojeg prati. Moguće je na drugi element postaviti da sidrište prati visinu odabranog elementa te će se trenutni element mijenjati relativno s obzirom na pridruženi. Na slici \ref{fig:sidriste} je moguće vidjeti tekstualni element kojemu je sidrište postavljeno da prati gornji desni kut roditelja. Na ovaj način će se tekst uvijek pomicati relativno sa gornjim desnim kutom.
\begin{figure}[h]
	\includegraphics[width=12.5cm, height=8cm]{sidriste.png}
	\centering
	\caption{Sidrište}
	\label{fig:sidriste}
\end{figure}
\newline
Sa desne strane na slici \ref{fig:sidriste} je moguće vidjeti sve komponente koje ima element te standardne opcije kao što su x, y i z transformacija unutar roditelja, visina i širina, minimalna i maksimalna vrijednost sidrišta i veličina. Tekstu možemo definirati oblik, veličinu, poravnavanje i boju.

\subsection{Tekst objekti}
Tokom igranja se može u gornjem desnom kutu vidjeti tekst, koji pokazuje na kojem smo trenutno krugu, u gornjem lijevom kutu bodove koje je igrač ostvario, a prilikom završetka igre se prikazuje tekst koji govori da je igrač uspješno odvozio i ispisuje ukupne bodove koje je igrač sakupio. Ovo su ustvari tri različite tekstualna objekta preko čije se reference može mijenjati tekst iz skripti. 

\subsubsection{Modovi izrade}
Postoje dvije glavne podjele ovih modova:
\begin{itemize} 
	\item Pozicija zaslona (\emph{eng.~Screen Space}) - platno se pozicionira prema poziciji zaslona. 
		\begin{itemize}
			\item Prekriti (\emph{eng.~Overlay}) - svi objekti se izrađuju na platnu te mijenjaju svoju veličinu ovisno o veličini platna
			\item Kamera - svi objekti se izrađuju ovisno o postavkama kamere. Ako se na kameri promijeni perspektiva, tada će se i sami izgled sučelja mijenjati ovisno o udaljenosti, kutu te svim ostalim parametrima koji su postavljeni.
		\end{itemize}
	\item Pozicija svijeta (\emph{eng.~World Space}) - platno se ponaša kao svaki drugi igrajući objekt. Moći će se postaviti u pozadini iza nekih drugih igrajućih objekata ili slično. Primjer bi bio prikaz teksta koji se mijenja na nekakvom računalu.
\end{itemize}

\subsection{Glavni izbornik}
Glavni izbornik (slika \ref{fig:glavniIzbornik}) se sastoji od tri igrajuća objekta, jednog teksta i dva botuna. Izborniku je pridružena i skripta koja kontrolira prikaz i funkcionalnost botuna. Koristi se ugrađena funkcionalnost Unity platforme za pozivanje metoda unutar skripte. Svaki botun ima okidač na klik te se samo treba postaviti metoda koja se želi pozvati. U igri su to dvije metode započni novu igru \texttt{QuitGame} i izađi iz igre \texttt{StartNewGame}. 
\begin{figure}[h]
	\includegraphics[width=12.5cm, height=8cm]{glavni_izbornik.png}
	\centering
	\caption{Glavni izbornik}
	\label{fig:glavniIzbornik}
\end{figure}

Pauziranje same igre se obavlja postavljanjem razlike vremena između slika \texttt{Time.deltaTime} na 0. Na ovaj način se ništa ne kreće i dobije se dojam da je igra pauzirana. Nova igra i izlazak iz igre se izvršavaju preko klase aplikacija (\emph{eng.~Application}) koja se isto može koristiti za pokretanje novog levela igre.