\section{Igrajući objekt}
Najvažniji objekti u unity-u koji sadrže modele, teren, likove i sve što se može vidjeti u sceni se zove \textbf{igrajući objekt} (\emph{eng. GameObject}). Objekti zasebno ne znače puno bez komponenti koje se nalaze u njima. Moglo bi se reći da je igrajući objekt ustvari samo kontenjer koji sadrži pojedine komponente. Svaki igrajući objekt mora imati samo jednu transformaciju (\emph{eng. Transform}) i pivot. Transformacija se koristi za mijenjanje veličine, pozicije i rotacije objekta. Pivot definira oko čega će se rotirati i mijenjati svoju veličinu, te ne utječe na kretanje modela po x, y i z osima.
\vspace{2mm}
\newline Vozilo korišteno kroz cijelu igru je također igrajući objekt, koji je izrađen u alatu \textbf{SketchUp}. Složeniji objekti se izrađuju u alatima za 3D modeliranje, te kasnije samo unesu u unity jer unity ne pruža mogućnost izrade naprednijih modela. 

