\section{Korištene tehnologije}
\subsection{Unity}
Unity je jedan od najpopularnijih razvojnih platformi za izradu 2D i 3D igrica. Moguće je korištenjem ovog alata napraviti jednu verziju igrice koja će se moći pokretati na računalu, igrajućim konzolama, mobilnim uređajima i web stranicama. \par
Skripte se mogu pisati u C\# ili javascriptu. Preporuka je pisati u C\# zbog samog stila pisanja jer se u kratkom periodu dosegne više od pedeset linija k\^oda i koristi dosta ugrađenih metoda. Za sve ugrađene metoda treba znati koji su argumenti koje primaju, a ako se koristi javascript to se neće moći vidjeti. Framework koji se koristi za pisanje je Monodevelop jer se unity može koristiti na Windows mašinama i na Macintosh mašinama. \par
Moguće je besplatno koristit unity dok se ne dosegne prihod od 100,000 dolara. Ukoliko se zaradi ova svota novca preko igre razvijene u unity-u, tada je potrebno kupiti profesionalnu verziju. Trenutna verzija je 5.0, koja nažalost još uvijek ima problema sa linux platformom.

\subsection{SketchUp}
SketchUp je alat za izradu 3D geometrijskih tijela napravljen od strane Google-a. Koristi se većinom u građevinskim obrtima zbog vrlo jednostavnog načina izrade modela. Dvodimenzionalni elementima se jednostavno dodaje treća dimenzija preko gurni/povuci (\emph{eng. Push/Pull}) alata. Ne preporuča se za izradu složenijih modela jer ne pruža dovoljno mogućnosti kao neki drugi alati. \par
Vozilo korišteno kroz cijelu igru je također izrađen u SketchUp-u. Nakon što se izrade kompletni modeli i njihove animacije, treba se napraviti datoteka koja ima 3ds ekstenziju.

