\section{Korištene tehnologije}
\subsection{Unity}
Unity je jedan od najpopularnijih razvojnih platformi za izradu 2D i 3D igrica. Moguće je korištenjem ovog alata napraviti jednu verziju igrice koja će se moći pokretati na računalu, igrajućim konzolama, mobilnim uređajima i web stranicama. Napravljen je kako bi svatko na svijetu imao priliku napraviti igru, te je pustiti na bilo koju platformu. Trenutno je vodeća platforma za izradu 3D mobilnih igrica.

Skripte se mogu pisati u C\# ili Unityscript. Preporuka iskusnijih programera je pisati u C\# zbog lakšeg provjeravanja pogrešaka u k\^odu. U Unityscriptu nije potrebno definirati tip varijable, pa je puno teže u većim skriptama pronaći razlog pogreške. Okvir \emph{eng.~Framework} koji se koristi za pisanje je Monodevelop jer se unity može koristiti na Windows mašinama i na Macintosh mašinama. 

Moguće je besplatno koristiti Unity dok se ne dosegne prihod od 100,000 dolara. Ukoliko se zaradi ova svota novca preko igre razvijene u Unity platformi, tada je potrebno kupiti profesionalnu verziju. Trenutna verzija je 5.0, koja nažalost još uvijek ima problema sa Linux platformom.

\subsection{SketchUp}
SketchUp je alat za izradu 3D geometrijskih tijela koji je napravio Google. Koristi se većinom u građevinskim obrtima zbog vrlo jednostavnog načina izrade modela. Dvodimenzionalnim elementima se jednostavno dodaje treća dimenzija preko gurni/povuci (\emph{eng.~Push/Pull}) alata. Ne preporuča se za izradu složenijih modela jer ne pruža dovoljno mogućnosti kao neki drugi alati. 

Vozilo korišteno kroz cijelu igru je također izrađeno u SketchUp-u. Nakon što se izrade kompletni modeli i njihove animacije, treba se napraviti datoteka koja ima 3ds ekstenziju.

