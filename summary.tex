{ \fontsize{16pt}{12pt}\selectfont\bfseries Sažetak}\\
U ovom radu je prikazan proces izrade trodimenzionalne igre korištenjem Unity platforme. Tip igre koji je napravljen je utrka, gdje automobil skuplja bodove i treba proći dva kruga. Igra nakon izrade se može igrati na skoro svim platformama (android, Windows, Mac OSX,  WebGL, itd.), znači na onim platformama na koje se proizvedu izvršne datoteke.  Kako za igru trebaju modeli, izrađeni su korištenjem tehnologije SketchUp te se nakon izrade uključe u Unity. Automobil posjeduje sve osnovne kretnje te je stvoren ambijent za zanimljivije iskustvo tokom igranja. Konačna igra je skup svih pojedinih dijelova kao što su zvukovi, modeli i skripte koji čine jednu jedinstvenu cjelinu.\par
\vspace{10mm}
{ \fontsize{16pt}{14pt}\selectfont\bfseries Summary}\par
{ \fontsize{14pt}{12pt}\selectfont Development of a 3D game in Unity}\\
In this paper the process of developing a three dimensional game is shown using the Unity platform. The game type is a race game, where a car collects points and needs to make two laps. After the creation of the game it can be played on almost any platform (android, Windows, Mac OSX, etc.), meaning on those platforms for which we make the executable files. Because the game needs models, they are built using the SketchUp technology, which are then imported in Unity. The car has all of the basic movements and the ambient is made for a better user experiance. The final game is a collection of sounds, models and scripts, which make a whole.
